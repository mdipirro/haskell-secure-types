
%% bare_conf.tex
%% V1.4b
%% 2015/08/26
%% by Michael Shell
%% See:
%% http://www.michaelshell.org/
%% for current contact information.
%%
%% This is a skeleton file demonstrating the use of IEEEtran.cls
%% (requires IEEEtran.cls version 1.8b or later) with an IEEE
%% conference paper.
%%
%% Support sites:
%% http://www.michaelshell.org/tex/ieeetran/
%% http://www.ctan.org/pkg/ieeetran
%% and
%% http://www.ieee.org/

%%*************************************************************************
%% Legal Notice:
%% This code is offered as-is without any warranty either expressed or
%% implied; without even the implied warranty of MERCHANTABILITY or
%% FITNESS FOR A PARTICULAR PURPOSE! 
%% User assumes all risk.
%% In no event shall the IEEE or any contributor to this code be liable for
%% any damages or losses, including, but not limited to, incidental,
%% consequential, or any other damages, resulting from the use or misuse
%% of any information contained here.
%%
%% All comments are the opinions of their respective authors and are not
%% necessarily endorsed by the IEEE.
%%
%% This work is distributed under the LaTeX Project Public License (LPPL)
%% ( http://www.latex-project.org/ ) version 1.3, and may be freely used,
%% distributed and modified. A copy of the LPPL, version 1.3, is included
%% in the base LaTeX documentation of all distributions of LaTeX released
%% 2003/12/01 or later.
%% Retain all contribution notices and credits.
%% ** Modified files should be clearly indicated as such, including  **
%% ** renaming them and changing author support contact information. **
%%*************************************************************************


% *** Authors should verify (and, if needed, correct) their LaTeX system  ***
% *** with the testflow diagnostic prior to trusting their LaTeX platform ***
% *** with production work. The IEEE's font choices and paper sizes can   ***
% *** trigger bugs that do not appear when using other class files.       ***                          ***
% The testflow support page is at:
% http://www.michaelshell.org/tex/testflow/



\documentclass[conference]{IEEEtran}
% Some Computer Society conferences also require the compsoc mode option,
% but others use the standard conference format.
%
% If IEEEtran.cls has not been installed into the LaTeX system files,
% manually specify the path to it like:
% \documentclass[conference]{../sty/IEEEtran}





% Some very useful LaTeX packages include:
% (uncomment the ones you want to load)


% *** MISC UTILITY PACKAGES ***
%
%\usepackage{ifpdf}
% Heiko Oberdiek's ifpdf.sty is very useful if you need conditional
% compilation based on whether the output is pdf or dvi.
% usage:
% \ifpdf
%   % pdf code
% \else
%   % dvi code
% \fi
% The latest version of ifpdf.sty can be obtained from:
% http://www.ctan.org/pkg/ifpdf
% Also, note that IEEEtran.cls V1.7 and later provides a builtin
% \ifCLASSINFOpdf conditional that works the same way.
% When switching from latex to pdflatex and vice-versa, the compiler may
% have to be run twice to clear warning/error messages.





\usepackage[numbers]{natbib}
% *** CITATION PACKAGES ***
%
%\usepackage{cite}
% cite.sty was written by Donald Arseneau
% V1.6 and later of IEEEtran pre-defines the format of the cite.sty package
% \cite{} output to follow that of the IEEE. Loading the cite package will
% result in citation numbers being automatically sorted and properly
% "compressed/ranged". e.g., [1], [9], [2], [7], [5], [6] without using
% cite.sty will become [1], [2], [5]--[7], [9] using cite.sty. cite.sty's
% \cite will automatically add leading space, if needed. Use cite.sty's
% noadjust option (cite.sty V3.8 and later) if you want to turn this off
% such as if a citation ever needs to be enclosed in parenthesis.
% cite.sty is already installed on most LaTeX systems. Be sure and use
% version 5.0 (2009-03-20) and later if using hyperref.sty.
% The latest version can be obtained at:
% http://www.ctan.org/pkg/cite
% The documentation is contained in the cite.sty file itself.






% *** GRAPHICS RELATED PACKAGES ***
%
\ifCLASSINFOpdf
  % \usepackage[pdftex]{graphicx}
  % declare the path(s) where your graphic files are
  % \graphicspath{{../pdf/}{../jpeg/}}
  % and their extensions so you won't have to specify these with
  % every instance of \includegraphics
  % \DeclareGraphicsExtensions{.pdf,.jpeg,.png}
\else
  % or other class option (dvipsone, dvipdf, if not using dvips). graphicx
  % will default to the driver specified in the system graphics.cfg if no
  % driver is specified.
  % \usepackage[dvips]{graphicx}
  % declare the path(s) where your graphic files are
  % \graphicspath{{../eps/}}
  % and their extensions so you won't have to specify these with
  % every instance of \includegraphics
  % \DeclareGraphicsExtensions{.eps}
\fi
% graphicx was written by David Carlisle and Sebastian Rahtz. It is
% required if you want graphics, photos, etc. graphicx.sty is already
% installed on most LaTeX systems. The latest version and documentation
% can be obtained at: 
% http://www.ctan.org/pkg/graphicx
% Another good source of documentation is "Using Imported Graphics in
% LaTeX2e" by Keith Reckdahl which can be found at:
% http://www.ctan.org/pkg/epslatex
%
% latex, and pdflatex in dvi mode, support graphics in encapsulated
% postscript (.eps) format. pdflatex in pdf mode supports graphics
% in .pdf, .jpeg, .png and .mps (metapost) formats. Users should ensure
% that all non-photo figures use a vector format (.eps, .pdf, .mps) and
% not a bitmapped formats (.jpeg, .png). The IEEE frowns on bitmapped formats
% which can result in "jaggedy"/blurry rendering of lines and letters as
% well as large increases in file sizes.
%
% You can find documentation about the pdfTeX application at:
% http://www.tug.org/applications/pdftex





% *** MATH PACKAGES ***
%
%\usepackage{amsmath}
% A popular package from the American Mathematical Society that provides
% many useful and powerful commands for dealing with mathematics.
%
% Note that the amsmath package sets \interdisplaylinepenalty to 10000
% thus preventing page breaks from occurring within multiline equations. Use:
%\interdisplaylinepenalty=2500
% after loading amsmath to restore such page breaks as IEEEtran.cls normally
% does. amsmath.sty is already installed on most LaTeX systems. The latest
% version and documentation can be obtained at:
% http://www.ctan.org/pkg/amsmath





% *** SPECIALIZED LIST PACKAGES ***
%
%\usepackage{algorithmic}
% algorithmic.sty was written by Peter Williams and Rogerio Brito.
% This package provides an algorithmic environment fo describing algorithms.
% You can use the algorithmic environment in-text or within a figure
% environment to provide for a floating algorithm. Do NOT use the algorithm
% floating environment provided by algorithm.sty (by the same authors) or
% algorithm2e.sty (by Christophe Fiorio) as the IEEE does not use dedicated
% algorithm float types and packages that provide these will not provide
% correct IEEE style captions. The latest version and documentation of
% algorithmic.sty can be obtained at:
% http://www.ctan.org/pkg/algorithms
% Also of interest may be the (relatively newer and more customizable)
% algorithmicx.sty package by Szasz Janos:
% http://www.ctan.org/pkg/algorithmicx




% *** ALIGNMENT PACKAGES ***
%
%\usepackage{array}
% Frank Mittelbach's and David Carlisle's array.sty patches and improves
% the standard LaTeX2e array and tabular environments to provide better
% appearance and additional user controls. As the default LaTeX2e table
% generation code is lacking to the point of almost being broken with
% respect to the quality of the end results, all users are strongly
% advised to use an enhanced (at the very least that provided by array.sty)
% set of table tools. array.sty is already installed on most systems. The
% latest version and documentation can be obtained at:
% http://www.ctan.org/pkg/array


% IEEEtran contains the IEEEeqnarray family of commands that can be used to
% generate multiline equations as well as matrices, tables, etc., of high
% quality.




% *** SUBFIGURE PACKAGES ***
%\ifCLASSOPTIONcompsoc
%  \usepackage[caption=false,font=normalsize,labelfont=sf,textfont=sf]{subfig}
%\else
%  \usepackage[caption=false,font=footnotesize]{subfig}
%\fi
% subfig.sty, written by Steven Douglas Cochran, is the modern replacement
% for subfigure.sty, the latter of which is no longer maintained and is
% incompatible with some LaTeX packages including fixltx2e. However,
% subfig.sty requires and automatically loads Axel Sommerfeldt's caption.sty
% which will override IEEEtran.cls' handling of captions and this will result
% in non-IEEE style figure/table captions. To prevent this problem, be sure
% and invoke subfig.sty's "caption=false" package option (available since
% subfig.sty version 1.3, 2005/06/28) as this is will preserve IEEEtran.cls
% handling of captions.
% Note that the Computer Society format requires a larger sans serif font
% than the serif footnote size font used in traditional IEEE formatting
% and thus the need to invoke different subfig.sty package options depending
% on whether compsoc mode has been enabled.
%
% The latest version and documentation of subfig.sty can be obtained at:
% http://www.ctan.org/pkg/subfig




% *** FLOAT PACKAGES ***
%
%\usepackage{fixltx2e}
% fixltx2e, the successor to the earlier fix2col.sty, was written by
% Frank Mittelbach and David Carlisle. This package corrects a few problems
% in the LaTeX2e kernel, the most notable of which is that in current
% LaTeX2e releases, the ordering of single and double column floats is not
% guaranteed to be preserved. Thus, an unpatched LaTeX2e can allow a
% single column figure to be placed prior to an earlier double column
% figure.
% Be aware that LaTeX2e kernels dated 2015 and later have fixltx2e.sty's
% corrections already built into the system in which case a warning will
% be issued if an attempt is made to load fixltx2e.sty as it is no longer
% needed.
% The latest version and documentation can be found at:
% http://www.ctan.org/pkg/fixltx2e


%\usepackage{stfloats}
% stfloats.sty was written by Sigitas Tolusis. This package gives LaTeX2e
% the ability to do double column floats at the bottom of the page as well
% as the top. (e.g., "\begin{figure*}[!b]" is not normally possible in
% LaTeX2e). It also provides a command:
%\fnbelowfloat
% to enable the placement of footnotes below bottom floats (the standard
% LaTeX2e kernel puts them above bottom floats). This is an invasive package
% which rewrites many portions of the LaTeX2e float routines. It may not work
% with other packages that modify the LaTeX2e float routines. The latest
% version and documentation can be obtained at:
% http://www.ctan.org/pkg/stfloats
% Do not use the stfloats baselinefloat ability as the IEEE does not allow
% \baselineskip to stretch. Authors submitting work to the IEEE should note
% that the IEEE rarely uses double column equations and that authors should try
% to avoid such use. Do not be tempted to use the cuted.sty or midfloat.sty
% packages (also by Sigitas Tolusis) as the IEEE does not format its papers in
% such ways.
% Do not attempt to use stfloats with fixltx2e as they are incompatible.
% Instead, use Morten Hogholm'a dblfloatfix which combines the features
% of both fixltx2e and stfloats:
%
% \usepackage{dblfloatfix}
% The latest version can be found at:
% http://www.ctan.org/pkg/dblfloatfix




% *** PDF, URL AND HYPERLINK PACKAGES ***
%
%\usepackage{url}
% url.sty was written by Donald Arseneau. It provides better support for
% handling and breaking URLs. url.sty is already installed on most LaTeX
% systems. The latest version and documentation can be obtained at:
% http://www.ctan.org/pkg/url
% Basically, \url{my_url_here}.




% *** Do not adjust lengths that control margins, column widths, etc. ***
% *** Do not use packages that alter fonts (such as pslatex).         ***
% There should be no need to do such things with IEEEtran.cls V1.6 and later.
% (Unless specifically asked to do so by the journal or conference you plan
% to submit to, of course. )


\usepackage{bussproofs}
\usepackage{listings}

% correct bad hyphenation here
\hyphenation{op-tical net-works semi-conduc-tor}


\begin{document}
%
% paper title
% Titles are generally capitalized except for words such as a, an, and, as,
% at, but, by, for, in, nor, of, on, or, the, to and up, which are usually
% not capitalized unless they are the first or last word of the title.
% Linebreaks \\ can be used within to get better formatting as desired.
% Do not put math or special symbols in the title.
\title{Ensuring Information Security by Using\\Haskell Advanced Type System}


% author names and affiliations
% use a multiple column layout for up to three different
% affiliations
\author{\IEEEauthorblockN{Matteo Di Pirro}
\IEEEauthorblockA{Department of Mathematics\\
University of Padua\\
Padua, Italy}}
% conference papers do not typically use \thanks and this command
% is locked out in conference mode. If really needed, such as for
% the acknowledgment of grants, issue a \IEEEoverridecommandlockouts
% after \documentclass

% for over three affiliations, or if they all won't fit within the width
% of the page, use this alternative format:
% 
%\author{\IEEEauthorblockN{Michael Shell\IEEEauthorrefmark{1},
%Homer Simpson\IEEEauthorrefmark{2},
%James Kirk\IEEEauthorrefmark{3}, 
%Montgomery Scott\IEEEauthorrefmark{3} and
%Eldon Tyrell\IEEEauthorrefmark{4}}
%\IEEEauthorblockA{\IEEEauthorrefmark{1}School of Electrical and Computer Engineering\\
%Georgia Institute of Technology,
%Atlanta, Georgia 30332--0250\\ Email: see http://www.michaelshell.org/contact.html}
%\IEEEauthorblockA{\IEEEauthorrefmark{2}Twentieth Century Fox, Springfield, USA\\
%Email: homer@thesimpsons.com}
%\IEEEauthorblockA{\IEEEauthorrefmark{3}Starfleet Academy, San Francisco, California 96678-2391\\
%Telephone: (800) 555--1212, Fax: (888) 555--1212}
%\IEEEauthorblockA{\IEEEauthorrefmark{4}Tyrell Inc., 123 Replicant Street, Los Angeles, California 90210--4321}}




% use for special paper notices
%\IEEEspecialpapernotice{(Invited Paper)}




% make the title area
\maketitle

% As a general rule, do not put math, special symbols or citations
% in the abstract
\begin{abstract}
Protecting data confidentiality and working on validated values have become increasingly important in modern software. Since the first examples of code injection or buffer overflow attacks, language-based security has improved itself. We have developed a huge amount of techniques for detecting software threats. We have also developed from scratch entire new languages for security purposes. \\
The core idea behind that is to ensure properties by using type theory. With advanced and complex type systems we are able to check for code vulnerabilities at compile time. Nevertheless, learning a new language, or completely migrate to that, is a complex and difficult operation. This is why, over the last few years, researchers have developed new secure libraries for existing languages. \\
The contribute of this paper is twofold. On one hand I lightly build an existing Haskell library up. On the other hand I present two new types for ensuring two important properties: operating on validated input values and performing computations on untainted data.
\end{abstract}

% no keywords




% For peer review papers, you can put extra information on the cover
% page as needed:
% \ifCLASSOPTIONpeerreview
% \begin{center} \bfseries EDICS Category: 3-BBND \end{center}
% \fi
%
% For peerreview papers, this IEEEtran command inserts a page break and
% creates the second title. It will be ignored for other modes.
\IEEEpeerreviewmaketitle

\section{Introduction}
Over the last few years software has become increasingly complex. It is so much complex that is almost impossible to see how it can be abused. The problem is even worst when one is forced to trust other people's code. Many secure languages have been developed from scratch for solving this kinds of problems. Two well-known examples are Jif \cite{pullicino2014jif} by \citeauthor{pullicino2014jif} and Flowcaml \cite{simonet2003flow} by \citeauthor{simonet2003flow}. The former is a Java extension adding support for security labels such that the developers can specify confidentiality and integrity policies to the various variables used in their program. The letter, instead, is an extension of the Objective Caml language with a type system tracing information flow. Its purpose is basically to allow to write real programs and to automatically check whether they obey some security policy. \\
However, it is a very heavy-weight solution to introduce a new programming language. In fact, despite of the large work on that, there has been relatively little adoption of the proposed techniques. Moreover, often only a small part of the system (maybe only a few variables in a large program) has security requirements. This is why many researchers have developed new lightweight libraries for ensuring security properties while programming. This paper aims to go further this direction. \\
Here I present a Haskell based library ensuring some security properties and a real-world use case for them. This library might be used in scenarios where we want to incorporate in our programs some code written by outsiders (untrusted programmers) to access our private information. We would like to have a guarantee the program will not send our private data to an attacker. A slightly different, but related, scenario is where we ourselves write the possible unsafe code, but we want to have the help of the type checker to find possible security mistakes. \\
\citeauthor{li2006encoding} \cite{li2006encoding} have previously explored the possibility to ensure information flow security as a library, but their approach is \textit{arrow} based (\cite{hughes2000generalising}). Hence, programmers have to be familiar with arrows. Afterwards, \citeauthor{russo2008library} \cite{russo2008library} have shown that a monadic solution is also possible. Unfortunately, their work makes an intense use of the \texttt{IO} monad, thus their system is non-completely static. \\
In this paper I lightly build the last mentioned library up and formalise two new types. As a result, the following security properties are met by my work: secure information flow, augmented with \textit{declassification policies}, secure computation on untainted data and dynamic validation of user input before their use. \\
The remainder of this paper is organized as follows. Section~\ref{sec:assumptions} formalises my assumptions. Section~\ref{sec:example} describes the motivating example. Sections~\ref{sec:flow}, \ref{sec:computation} and \ref{sec:unsecure} provide descriptions and implementation details about the three secure types: \texttt{Unsecure}, \texttt{SecureFlow} and \texttt{SecureComputation}. Section~\ref{sec:limitations} sums my contribute up and emphasises its limitations. My discussion is then concluded in Section~\ref{sec:conclusion}.
\section{A real-world use case}\label{sec:example}
Consider a company which would like to manage its employees data and the situation of its stores. Suppose that every operation on those data must be done after a login. In such a scenario, passwords should remain secrets, and a declassification policy should be applied during the login procedure. In fact, after a successful login, a user knows that the provided password was correct. Then, the logged user may perform some operations on the stores data. Moreover, if he or she is a company leader, he or she can increase an employee's salary. The salary is strictly confidential, so nobody may know it. \\
The following security properties must be met in this application. First, passwords and salaries are confidential and they may not be shown in a public output channel (i.e.the console). Second, a natural number (e.g. the increment amount) coming from the user must be validated before its use. Moreover, in sensitive operations, the value must be marked as pure (i.e. untainted). \\
We would be sure that two of them, secure information flow and secure computation over data, are ensured at compile time, before the execution of the application. The third, input validation, must be performed at run-time, since inputs vary from an execution to another. Thus, the type system will only be able to check that an input value is validated before a real use. I will go deeper on that in section 5. \\
The application database is composed by three JSON (\textbf{J}ava\textbf{S}cript \textbf{O}bject \textbf{N}otation) files, as follows: \texttt{credentials.json}, \texttt{employees.json} and \texttt{stores.json}, with the obvious meaning. The choice of JSON files is motivated by the simplicity in their manipulation. A SQL-based (or whatever) version would be possible too.

\section{The Unsecure module}\label{sec:usecure}
The \texttt{Unsecure} module makes sure that an input value will be validated before its use. Below is its definition.
\begin{lstlisting}[frame=single]
type ValidationFunctions a b
  = [a -> Maybe b]
data Unsecure a b =  
  Unsecure (ValidationFunctions a b, a)
\end{lstlisting}
Here, \texttt{a} and \texttt{b} represent \textit{every possible} type, as usual in Haskell type definitions. The former is the input type and the latter is an \textit{error} type. Using an error type is necessary, because the validation could fail. If it fails, a list of errors is returned. Following the Haskell error mechanism, every error in that list is represented as a constructor of the type \texttt{b}. 

%\begin{prooftree}
%	\AxiomC{}
%	\RightLabel{UPURE}
%	\UnaryInfC{(a, ValidationFunctions a b) $\rightarrow$ Unsecure a b}
%\end{prooftree}
%\begin{prooftree}
%	\AxiomC{a $rightarrow$ a'}
%	\RightLabel{UMAP}
%	\BinaryInfC{Unsecure a b $\rightarrow$ Unsecure a b}
%\end{prooftree}
\section{The SecureFlow module}\label{sec:flow}
Software usually manipulates information with different security policies. Password, for example, are sensitive data, while names or birth dates are not. During the execution sometimes we want to be sure that sensitive information don't flow to insecure or public output channels. On the other hand, we have to perform some operations on these reserved data. As an example consider a login procedure. After a successful login the user is sure that the provided password is correct. \\
The goal with \texttt{SecureFlow} is to ensure information flow security and declassification policies at compile-time, so that if a code is correctly compiled there are no security violations.

\subsection{Security lattice}
\texttt{SecureFlow} is based on a lattice structure, first introduced by \citeauthor{denning1976lattice} \cite{denning1976lattice}, and represented in this library as a type family.
\section{The SecureComputation module}\label{sec:computation}
The idea behind \texttt{SecureComputation} is exactly the same as \texttt{SecureFlow}. The difference relies on its aim. The former provides a way of working with pure data. Taint analysis (\cite{schwartz2010all}, \cite{newsome2005dynamic}) is a well-known technique for dynamic detecting software vulnerabilities. However, this dynamic dimension is not fully sound when applied to modern software. It is so much complex that a full testing process is practically infeasible. Moreover, like Dijkstra said, "program testing can be used to show the presence of bugs, but never to show their absence". That means we can not just trust dynamic analysis. We need a statical check. \\
My version of this statical check is \texttt{SecureComputation} shown in a shortened form in Listing~\ref{lst:securecomp}.
\begin{lstlisting}[caption={SecureComputation module}, label={lst:securecomp}, breaklines=true]
data SC m a = SC a

type family MustBePure m :: Constraint
data P = P
data T = T
type instance (MustBePure P) = ()

open :: MustBePure m => SC m a -> a
open (SC a) = a

smap :: (MustBePure m, MustBePure m') => (a -> b) -> SC m a -> SC m' b
smap f (SC a) = SC $ f a

spure :: a -> SC m a
spure = SC

sapp :: (MustBePure m, MustBePure m') => SC m' (a -> b) -> SC m a -> SC m' b
sapp (SC f) sc = smap f sc

sreturn :: a -> SecureComputation m a
sreturn = spure

sbind :: (MustBePure m, MustBePure m') => SC m a -> (a -> SC m' b) -> SC m' b
sbind (SC a) f = f a
\end{lstlisting}
As one can note, \texttt{SecureComputation} is based on the same type family method as \texttt{SecureFlow}. Basically, \texttt{P} and \texttt{T} are singleton types meaning \textit{Pure} and \textit{Tainted}. Naturally, only \texttt{P} is defined as an instance of the \texttt{MustBePure} constraint. \\
An \textit{SecureComputation} encapsulated value may be opened if and only if the \texttt{SecureComputation} holder is pure. Furthermore, computations on encapsulated values are allowed if and only if those value are pure (in the meaning that their containers are). \\
Again, \texttt{SecureComputation} is not a monad because of its type constraints. Make it a monad would be possible, but it is out of this paper scope. For the sake of simplicity I redefine functor, applicative and monad functions with another name (actually just adding \textit{s} as prefix) so that it might be used \textit{like} a monad. The meaning of those functions (\texttt{smap}, \texttt{spure}, \texttt{sapp} and \texttt{sbind}) is the usual one except for type constraints. Validity of a potential real instantiation would be also provable without considering type constraints. I give an example in the following proposition.
\begin{proposition}
\texttt{SecureComputation}, without type constraints, is a monad.
\end{proposition}
\begin{proof}
I am to prove the three monad laws. During the proof, \texttt{>>=} and \texttt{return} are supposed to be, respectively, \texttt{sbind} and \texttt{sreturn}. I use the original names for the sake of consistency.
\begin{enumerate}
	\item \textbf{Left identity}: \texttt{return a >>= f = f a}
		\begin{lstlisting}
(return a >>= f) = (spure a >>= f) = 
= ((SC a) >>= f) = f a
		\end{lstlisting}
	
	\item \textbf{Right identity}: \texttt{m >>= return = m}
		\begin{lstlisting}
((SC a) >>= return) = 
= (return a) = (spure a) = 
= (SC a) = m
		\end{lstlisting}
	\item \textbf{Associativity}: \texttt{(m >>= f) >>= g = m >>= ($\backslash$x -> f x >>= g)} 
		\begin{lstlisting}
(((SF a) >>= f) >>= g) = 
= ((f a) >>= g) = (g (f a))

(SF a >>= (\x -> f x >>= g)) =
= (\x -> f x >>= g) a = 
= ((f a) >>= g) = (g (f a))
		\end{lstlisting}
\end{enumerate}
\texttt{SecureComputation} satisfies the three monad laws, hence it is a monad.
\end{proof}
\texttt{SecureComputation} is useful as a type for user-provided values. Listing~\ref{lst:unpureNat} shows how an input could be encapsulated and marked as tainted. 
\begin{lstlisting}[caption={Tainted natural number},label={lst:unpureNat}, breaklines=true]
getUnpureNat :: IO (SecureComputation T String)
getUnpureNat = do n <- getLine
return $ spure n
\end{lstlisting}
Note that programmers can mark as tainted every value they want. They are also able to make differences among different input sources and values. That points \texttt{SecureComputation} flexibility out. Recall running example, and in particular operations on stores. Every stored product has a price and a number representing its stocks. Increments or decrements on those numbers are allowed only with pure values. Otherwise a type error must be detected. A suitable general function should have the following type signature: \texttt{SC P a -> String -> (a -> Store -> Store) -> SC P [Store] -> SC P [Store]}. The parameters have meanings as follows:
\begin{enumerate}
	\item modification value (\textit{v});
	\item product name (\textit{p});
	\item the real modification function (\textit{f}) (for instance, \texttt{modifyPrice});
	\item a list of stores (s).
\end{enumerate}
It returns a list of stores where \textit{p} has been modified according to \textit{f} based on \textit{v}. Note that \textit{v} and \textit{s} must be pure, while it doesn't matter for \textit{p}. A type error is detected every time an unpure (or tainted) value is provided supposing it to be a pure one. If the code compiles and \texttt{SecureComputation} is cleverly used, there are not operations on pure data starting from tainted one. 
% An example of a floating figure using the graphicx package.
% Note that \label must occur AFTER (or within) \caption.
% For figures, \caption should occur after the \includegraphics.
% Note that IEEEtran v1.7 and later has special internal code that
% is designed to preserve the operation of \label within \caption
% even when the captionsoff option is in effect. However, because
% of issues like this, it may be the safest practice to put all your
% \label just after \caption rather than within \caption{}.
%
% Reminder: the "draftcls" or "draftclsnofoot", not "draft", class
% option should be used if it is desired that the figures are to be
% displayed while in draft mode.
%
%\begin{figure}[!t]
%\centering
%\includegraphics[width=2.5in]{myfigure}
% where an .eps filename suffix will be assumed under latex, 
% and a .pdf suffix will be assumed for pdflatex; or what has been declared
% via \DeclareGraphicsExtensions.
%\caption{Simulation results for the network.}
%\label{fig_sim}
%\end{figure}

% Note that the IEEE typically puts floats only at the top, even when this
% results in a large percentage of a column being occupied by floats.


% An example of a double column floating figure using two subfigures.
% (The subfig.sty package must be loaded for this to work.)
% The subfigure \label commands are set within each subfloat command,
% and the \label for the overall figure must come after \caption.
% \hfil is used as a separator to get equal spacing.
% Watch out that the combined width of all the subfigures on a 
% line do not exceed the text width or a line break will occur.
%
%\begin{figure*}[!t]
%\centering
%\subfloat[Case I]{\includegraphics[width=2.5in]{box}%
%\label{fig_first_case}}
%\hfil
%\subfloat[Case II]{\includegraphics[width=2.5in]{box}%
%\label{fig_second_case}}
%\caption{Simulation results for the network.}
%\label{fig_sim}
%\end{figure*}
%
% Note that often IEEE papers with subfigures do not employ subfigure
% captions (using the optional argument to \subfloat[]), but instead will
% reference/describe all of them (a), (b), etc., within the main caption.
% Be aware that for subfig.sty to generate the (a), (b), etc., subfigure
% labels, the optional argument to \subfloat must be present. If a
% subcaption is not desired, just leave its contents blank,
% e.g., \subfloat[].


% An example of a floating table. Note that, for IEEE style tables, the
% \caption command should come BEFORE the table and, given that table
% captions serve much like titles, are usually capitalized except for words
% such as a, an, and, as, at, but, by, for, in, nor, of, on, or, the, to
% and up, which are usually not capitalized unless they are the first or
% last word of the caption. Table text will default to \footnotesize as
% the IEEE normally uses this smaller font for tables.
% The \label must come after \caption as always.
%
%\begin{table}[!t]
%% increase table row spacing, adjust to taste
%\renewcommand{\arraystretch}{1.3}
% if using array.sty, it might be a good idea to tweak the value of
% \extrarowheight as needed to properly center the text within the cells
%\caption{An Example of a Table}
%\label{table_example}
%\centering
%% Some packages, such as MDW tools, offer better commands for making tables
%% than the plain LaTeX2e tabular which is used here.
%\begin{tabular}{|c||c|}
%\hline
%One & Two\\
%\hline
%Three & Four\\
%\hline
%\end{tabular}
%\end{table}


% Note that the IEEE does not put floats in the very first column
% - or typically anywhere on the first page for that matter. Also,
% in-text middle ("here") positioning is typically not used, but it
% is allowed and encouraged for Computer Society conferences (but
% not Computer Society journals). Most IEEE journals/conferences use
% top floats exclusively. 
% Note that, LaTeX2e, unlike IEEE journals/conferences, places
% footnotes above bottom floats. This can be corrected via the
% \fnbelowfloat command of the stfloats package.

\section{Conclusions}\label{sec:conclusion}
Taking ideas from the literature, in this paper I have presented a simple library for information security in Haskell. I have formalised three new types satisfying three important principles: mandatory input validation, non-inference and computation on pure data. Two of them ensure the corresponding property in a static way so that it is satisfied if and only if the source code compiles. The first, contrariwise, may only be checked at run-time. \\
Besides, the library provides a simple way for formalising declassification policies. Considering it is wholly generalised, it might be adapted for satisfying almost every security requirement. \\
The actual Haskell implementation is partially based on monads, a widespread concept in functional programming. Although only one out of three types is a monad instance, the idea behind the other two is exactly the same. It would be possible to make them concrete monad instances, but that would require an effort out of the scope of this paper. \\
The library implementation and every example shown in this paper are publicly available in \cite{mdipirroGitHub}. 




% conference papers do not normally have an appendix

% use section* for acknowledgment
\section*{Acknowledgment}



% trigger a \newpage just before the given reference
% number - used to balance the columns on the last page
% adjust value as needed - may need to be readjusted if
% the document is modified later
%\IEEEtriggeratref{8}
% The "triggered" command can be changed if desired:
%\IEEEtriggercmd{\enlargethispage{-5in}}

% references section

% can use a bibliography generated by BibTeX as a .bbl file
% BibTeX documentation can be easily obtained at:
% http://mirror.ctan.org/biblio/bibtex/contrib/doc/
% The IEEEtran BibTeX style support page is at:
% http://www.michaelshell.org/tex/ieeetran/bibtex/
\bibliographystyle{bibtex/IEEEtranN}
% argument is your BibTeX string definitions and bibliography database(s)
\bibliography{bibtex/IEEEfull,essay}
%\nocite{*}
%
% <OR> manually copy in the resultant .bbl file
% set second argument of \begin to the number of references
% (used to reserve space for the reference number labels box)
%\begin{thebibliography}{1}
%
%\bibitem{IEEEhowto:kopka}
%H.~Kopka and P.~W. Daly, \emph{A Guide to \LaTeX}, 3rd~ed.\hskip 1em plus
%  0.5em minus 0.4em\relax Harlow, England: Addison-Wesley, 1999.
%
%\end{thebibliography}




% that's all folks
\end{document}


