\section{Introduction}
Over the last few years software has become increasingly complex. It is so much complex that is almost impossible to see how it can be abused. The problem is even worst when one is forced to trust other people's code. Many secure languages have been developed from scratch for solving this kinds of problems. Two well-known examples are Jif \cite{pullicino2014jif} by \citeauthor{pullicino2014jif} and Flowcaml \cite{simonet2003flow} by \citeauthor{simonet2003flow}. The former is a Java extension which adds support for security labels such that the developers can specify confidentiality and integrity policies to the various variables used in their program. The letter, instead, is an extension of the Objective Caml language with a type system tracing information flow. Its purpose is basically to allow to write real programs and to automatically check that they obey some security policy. \\
However, it is a very heavy-weight solution to introduce a new programming language for dealing with security policies. In fact, despite of the large work on that, there has been relatively little adoption of the proposed techniques. This has led to a new approach. Moreover, often only a small part of the system (maybe only a few variables in a large program) has security requirements. There is a language adoption threshold based on the ratio of security requirements to functionality requirements, and this threshold is very high. This is why many researchers have developed new lightweight libraries for ensuring security properties while programming. This paper aims to go further this direction. \\
