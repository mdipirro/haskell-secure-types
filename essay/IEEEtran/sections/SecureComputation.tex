\section{The SecureComputation module}\label{sec:computation}
The idea behind \texttt{SecureComputation} is exactly the same as \texttt{SecureFlow}. The difference relies on its aim. The former provides a way of working with pure data. \\
Taint analysis (\cite{schwartz2010all}, \cite{newsome2005dynamic}) is a well-known technique for dynamic detecting software vulnerabilities. However, this dynamic dimension is not fully sound when applied to modern software. It is so much complex that a full testing process is practically infeasible. Moreover, like Dijkstra said, "program testing can be used to show the presence of bugs, but never to show their absence". That means we cannot just trust dynamic analysis. We need a statical check. \\
My version of this statical check is \texttt{SecureComputation} shown in a shortened form in Listing~\ref{lst:securecomp}.
\begin{lstlisting}[caption={SecureComputation module}, label={lst:securecomp}, breaklines=true]
data SC m a = SC a

type family MustBePure m :: Constraint
data P = P
data T = T
type instance (MustBePure P) = ()

open :: MustBePure m => SC m a -> a
open (SC a) = a

smap :: (MustBePure m, MustBePure m') => (a -> b) -> SC m a -> SC m' b
smap f (SC a) = SC $ f a

spure :: a -> SC m a
spure = SC

sapp :: (MustBePure m, MustBePure m') => SC m' (a -> b) -> SC m a -> SC m' b
sapp (SC f) sc = smap f sc

sreturn :: a -> SecureComputation m a
sreturn = spure

sbind :: (MustBePure m, MustBePure m') => SC m a -> (a -> SC m' b) -> SC m' b
sbind (SC a) f = f a
\end{lstlisting}
As one can note, \texttt{SecureComputation} is based on the same type family method as \texttt{SecureFlow}. Basically, \texttt{P} and \texttt{T} are singleton types meaning \textit{Pure} and \textit{Tainted}. Naturally, only \texttt{P} is defined as an instance of the \texttt{MustBePure} constraint. \\
An \textit{SecureComputation} encapsulated value may be opened if and only if the \texttt{SecureComputation} holder is pure. Furthermore, computations on encapsulated values are allowed if and only if those values are pure (in the meaning that their containers are). \\
Again, \texttt{SecureComputation} is not a monad because of its type constraints. Making it a monad would be possible (\cite{Sculthorpe:13:ConstrainedMonad}), as stated in Section~\ref{sec:unsecure}, but it is out of this paper scope. For the sake of simplicity I redefine functor, applicative and monad functions with another name (actually just adding \textit{s} as prefix) so that it might be used \textit{like} a monad. The meaning of those functions (\texttt{smap}, \texttt{spure}, \texttt{sapp} and \texttt{sbind}) is the usual one except for type constraints. Validity of a potential real instantiation would be also provable without considering type constraints. An example is given by the following proposition.
\begin{proposition}
\texttt{SecureComputation}, without type constraints, is a monad.
\end{proposition}
\begin{proof}
I am to prove the three monad laws. During the proof, \texttt{>>=} and \texttt{return} are supposed to be, respectively, \texttt{sbind} and \texttt{sreturn}. I use the original names for the sake of consistency.
\begin{enumerate}
	\item \textbf{Left identity}: \texttt{return a >>= f = f a}
		\begin{lstlisting}
(return a >>= f) = (spure a >>= f) = 
= ((SC a) >>= f) = f a
		\end{lstlisting}
	
	\item \textbf{Right identity}: \texttt{m >>= return = m}
		\begin{lstlisting}
((SC a) >>= return) = 
= (return a) = (spure a) = 
= (SC a) = m
		\end{lstlisting}
	\item \textbf{Associativity}: \texttt{(m >>= f) >>= g = m >>= ($\backslash$x -> f x >>= g)} 
		\begin{lstlisting}
(((SF a) >>= f) >>= g) = 
= ((f a) >>= g) = (g (f a))

(SF a >>= (\x -> f x >>= g)) =
= (\x -> f x >>= g) a = 
= ((f a) >>= g) = (g (f a))
		\end{lstlisting}
\end{enumerate}
\texttt{SecureComputation} satisfies the three monad laws, hence it is a monad.
\end{proof}
\texttt{SecureComputation} is useful as a type for user-provided values. Listing~\ref{lst:unpureNat} shows how an input could be encapsulated and marked as tainted. Haskell, in fact, does not provide a function returning a \texttt{Num} (where \texttt{Num} is an abstract type class concretised by every type class representing a number, such as \texttt{Int} or \texttt{Float}). Thus one has to use \texttt{getLine}, which returns a \texttt{String}. In Section~\ref{sec:unsecure} I showed a way of validating this \texttt{String}. Here, contrariwise, no validation is performed on the user provided value; it is just marked as tainted (or not pure), so that it cannot be used as a parameter when a pure computation is required.
\begin{lstlisting}[caption={Tainted natural number},label={lst:unpureNat}, breaklines=true]
getUnpureNat :: IO (SecureComputation T String)
getUnpureNat = do n <- getLine
                  return $ spure n
\end{lstlisting}
Note that programmers can mark as tainted every value they want. They are also able to make differences among different input sources and values. That points \texttt{SecureComputation} flexibility out. For instance, recall the running example, and in particular the operations on stores. Every stored product has a price and a number representing its stocks. Increments or decrements on those numbers are allowed only with pure values. Otherwise a type error must be detected. A suitable general function should have the following type signature:\\
\texttt{SC P a -> String -> (a -> Store -> Store) -> SC P [Store] -> SC P [Store]}. \\
The parameters have meanings as follows:
\begin{enumerate}
	\item modification value (\textit{v});
	\item product name (\textit{p});
	\item the real modification function (\textit{f}) (for instance, \texttt{modifyPrice});
	\item a list of stores (\textit{s}).
\end{enumerate}
It returns a list of stores where \textit{p} has been modified according to \textit{f} based on \textit{v}. Note that \textit{v} and \textit{s} must be pure while it doesn't matter for \textit{p}. A type error is detected every time an unpure (or tainted) value is provided supposing it to be a pure one. If the code compiles and \texttt{SecureComputation} is cleverly used, there are not operations on pure data based on tainted one. 