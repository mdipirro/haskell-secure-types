\begin{proposition}
\texttt{SecureFlow}, as defined, is a monad.
\end{proposition}
\begin{proof}
We are to prove the three monad laws. 
\begin{enumerate}
	\item \textbf{Left identity}: \texttt{return a >>= f = f a}
		\begin{lstlisting}
(return a >>= f) = (pure a >>= f) = 
= ((Allowed a) >>= f) = f a
		\end{lstlisting}
	
	\item \textbf{Right identity}: \texttt{m >>= return = m}
		\begin{enumerate}
			\item \texttt{m = Allowed a}:
			\begin{lstlisting}
((Allowed a) >>= return) = 
= (return a) = (pure a) = 
= (Allowed a) = m
			\end{lstlisting}
			\item \texttt{m = Denied}:
			\begin{lstlisting}
(Denied >>= return) = Denied = m
			\end{lstlisting}
		\end{enumerate}
	
	\item \textbf{Associativity}: \texttt{(m >>= f) >>= g = m >>= ($\backslash$x -> f x >>= g)} 
		\begin{enumerate}
			\item \texttt{m = Allowed a}:
			\begin{lstlisting}
(((Allowed a) >>= f) >>= g) = 
= ((f a) >>= g) = (g (f a))

(Allowed a >>= (\x -> f x >>= g)) =
= (\x -> f x >>= g) a = 
= ((f a) >>= g) = (g (f a))
			\end{lstlisting}
			\item \texttt{m = Denied}:
			\begin{lstlisting}
((Denied >>= f) >>= g) = 
(Denied >>= g) = Denied =
(Denied >>= (\x -> f x >>= g))
			\end{lstlisting}
		\end{enumerate}
\end{enumerate}
\texttt{SecureFlow} satisfies the three monad laws, hence it is a monad.
\end{proof}