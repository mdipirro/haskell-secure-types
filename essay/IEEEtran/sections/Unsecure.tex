\section{The Unsecure module}\label{sec:usecure}
The \texttt{Unsecure} module makes sure that an input value will be validated before its use. Below is its definition.
\begin{lstlisting}[frame=single]
type ValidationFunctions a b
  = [a -> Maybe b]
data Unsecure a b =  
  Unsecure (ValidationFunctions a b, a)
\end{lstlisting}
Here, \texttt{a} and \texttt{b} represent \textit{every possible} type, as usual in Haskell type definitions. The former is the input type and the latter is an \textit{error} type. Using an error type is necessary, because the validation could fail. If it fails, a list of errors is returned. Following the Haskell error mechanism, every error in that list is represented as a constructor of the type \texttt{b}. 

%\begin{prooftree}
%	\AxiomC{}
%	\RightLabel{UPURE}
%	\UnaryInfC{(a, ValidationFunctions a b) $\rightarrow$ Unsecure a b}
%\end{prooftree}
%\begin{prooftree}
%	\AxiomC{a $rightarrow$ a'}
%	\RightLabel{UMAP}
%	\BinaryInfC{Unsecure a b $\rightarrow$ Unsecure a b}
%\end{prooftree}