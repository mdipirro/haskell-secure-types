\section{The Unsecure module}\label{sec:usecure}
The \texttt{Unsecure} module makes sure an input value will be validated before its use. Listing~\ref{lst:unsecure} shows its definition.

\begin{lstlisting}[caption={Unsecure module}, label={lst:unsecure}]
type ValidationFunctions a b =
  [a -> Maybe b]
data Unsecure a b =  
  Unsecure (ValidationFunctions a b, a)
 
validate :: Unsecure a b -> Either a [b]

umap :: Unsecure a b -> (a -> a) 
  -> Unsecure a b

upure :: a -> ValidationFunctions a b
  -> Unsecure a b
\end{lstlisting}
Here, \texttt{a} and \texttt{b} represent \textit{every possible} type, as usual in Haskell type definitions. The former is the input type and the latter is an \textit{error} type. Using an error type is necessary, because the validation could fail. If it fails, a list of errors is returned. Following the Haskell error mechanism, every error in that list is represented as a constructor of the type \texttt{b}. \\
\texttt{Unsecure} is defined as a pair of \texttt{ValidationFunctions} (\textit{VF}), from value of type \texttt{a} to a value of an error type \texttt{b}, and a value of type \texttt{a} (the ''value''). \texttt{validate} simply makes sure that the ''value'' actually meets the \textit{VF} constraints. If so, it returns the ''value'', otherwise it returns the error list, since the ''value'' could fail more than one constraint. This is why \texttt{Either} is required. \\
\texttt{upure} and \texttt{umap} try to simulate an applicative approach. The former allows programmers to create an \texttt{Unsecure} value. The latter allows them to manipulate the ''value'' before validation. An \texttt{Unsecure} value is supposed to be created by an IO function and manipulated by programmers. It may not be defined as a functor or an applicative one because of its own nature. Type \texttt{a} must not be changed, because with this definition the validation functions are based on it. A possible solutions would be using a constrained functor in order to be sure that the ''value'' type could be constrained. However, this may not be done with normal functors. Nevertheless, \citeauthor{Sculthorpe:13:ConstrainedMonad} \cite{Sculthorpe:13:ConstrainedMonad} had shown that this is possible with a little effort, and there are a lot of other works on this topic (an example is Co-Yoneda functor, based on Yoneda Lemma, \cite{elkins2009calculating}, \cite{yoneda}). For the sake of simplicity, this work is limited to classic functor, applicative ones and monads. \\
\begin{lstlisting}[label={unsecureNat}, breaklines=true, caption={Unsecure for natural numbers}]
data NatError = NegativeNumber | NonNumeric

getNat :: IO (Unsecure String NatError)
getNat = do n <- getLine
	    return $ upure n [
		isNumeric,
		isNatural
	    ]	   
	    
isNumeric s = if null $ dropWhile isDigit s
              then Nothing
              else Just NonNumeric

isNatural n = if read n >= 0
              then Nothing
              else Just NegativeNumber
              
useNat = do n <- getNat
            case validate n of
              Left nat -> operation nat
              Right es -> showerrors es
\end{lstlisting}
Listing~\ref{unsecureNat} shows how \texttt{Unsecure} can be used for validating strings as natural numbers. \texttt{getNat} takes a string from the user and constraint that with two functions, \texttt{isNumeric} and \texttt{isNatural}, with the obvious meaning. \texttt{NatError} represents the category of possible errors. Each validation function should have a respective error type. When a user is asked for a natural number an \texttt{IO (Unsecure String NatError)} is given to the programmers instead of an \texttt{IO Int}. Doing so, they can be sure that the boxed \texttt{String} actually represents what they want. \\
Even with this minimal example the \texttt{Unsecure} benefits are clear: programmers must validate inputs before using them. As an immediate consequence, many vulnerabilities are called off. AS another example, think to a function which return a string and validates it by checking his length. Buffer overflow attacks are based on a poor check of the string length, so by using \texttt{Unsecure} we are sure about the validity of that string. This is possible with a few lines of code. 